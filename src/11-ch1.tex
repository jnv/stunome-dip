\chapter{Přehled hypertextových systémů}

Tato kapitola představuje některé významné hypertextové systémy či jejich koncepty. Většina systémů vznikla před existencí Webu a nabízí zcela odlišný přístup k práci s hypertextem, který se spíše přibližuje koncepci Xanadu. V přehledu není zahrnutý systém Xanadu, kterému je věnovaná následující kapitola, ani některé populární systémy, jako například Guide nebo HyperCard. Tyto systémy sice rozšířily myšlenku hypertextu mezi laickou veřejností, ale jejich model práce s hypertextem je spíše omezený a v porovnání s ostatními systémy nepřináší mnoho nových nápadů. Pro rozsáhlejší přehled systémů a jejich funkcí viz \textcites{Conklin1987}{Nielsen1995}{Muller-Prove2002}.

\section{Traité de documentation}

Řada autorů hypertextových systémů našla inspiraci v Memexu Vannevara Bushe. Avšak půl století před Bushem řešil belgický knihovník Paul Otlet fundamentální problémy se systemizací literatury. Podle Otleta je „vazba knihy, její formát i osoba autora nedůležitá za předpokladu, že se substance díla, jeho zdroje a závěry stanou součástí kolektivní organizace vědomostí“ \autocite[237]{Rayward1994}. Většina organizačních systémů zavede čtenáře až ke knize -- ale ne dále.
%\autocite[186--187]{Wright2007}
Otlet proto navrhuje každou knihu systematicky analyzovat po kapitolách ve čtyřech oblastech: fakta, interpretace fakt, statistiky a zdroje, což by umožnilo zjistit její přínosy pro celkové vědění. Následně by „kusy“ informací byly uloženy a zařazeny na kartičkách. Vedle toho je zapotřebí udržovat podrobný a zároveň kompletní přehled znalostí. To Otleta a jeho kolegy vedlo k rozšíření a internacionalizaci Deweyho desetinné soustavy na fasetový systém Mezinárodního desetinného třídění \autocite[238]{Rayward1994}. Fasety umožňují dokument zatřídit podle jazyka, místa, času a „fyzických charakteristik,“ případně vytvářet vazby mezi tematickými třídami, což může pomoci k pochopení „sociálního kontextu“ dokumentu \autocite[188]{Wright2007}.

V rámci Světové výstavy EXPO 1910 která se konala v Bruselu vytvořil Otlet s La Fontainem instalaci \textit{Mundaneum} jako utopickou vizi kosmopolitního „města intelektu.“ Za podpory belgické vlády se Otletův plán začal formovat do podoby enormní kolekce kartiček a dokumentů, tehdy nejmodernější technologie pro skladování informací. S iniciativou byla spojená i vyhledávací služba – uživatelé mohli poštou zaslat dotazy a za poplatek 27 franků na tisíc kartiček obdrželi odpověď. \autocite[188--189]{Wright2007}

Otletovo stěžejní dílo \textit{Traité de Documentation} z roku 1934 pak posouvá vizi práce s informacemi mnohem dále. Otlet si byl dobře vědom omezení fyzické formy dokumentů; jeho vize popisuje pracovní stanici kde již nejsou žádné knihy:

\begin{quoted}{\autocite[Cit. podle][185]{Wright2007}}
Namísto knih je na dosah obrazovka a telefon. Jinde, v ohromné budově, jsou všechny knihy a informace. Odtud se stránka ke~čtení (…) zobrazí na obrazovce. Obrazovka může být rozdělená na poloviny, čtvrtiny nebo i desetiny, pokud je zapotřebí procházet více dokumentů současně. (…) Biograf, fonograf, rádio, televize: tyto nástroje, považované za náhradu knihy, se ve skutečnosti stanou novou knihou, nejmocnějším dílem pro rozšíření lidského vědění. Toto bude rádiová knihovna a televizní kniha.
\end{quoted}

Nové prostředky komunikace, jako rádio a televize povedou ke vzniku „poslechové dokumentace.“ Otlet předpokládal, že dojde i na další smysly, vzniknout hmatové, chuťové a čichové dokumenty, které se budou vzájemně doplňovat \autocite[244]{Rayward1994}.

Ve spolupráci s „doplňkovými organizacemi“ jako jsou univerzity, výzkumné instituce, mezinárodní firmy by skrze systematickou a standardizovanou organizaci dokumentů a redistribuci informací mohla vzniknout celosvětová Univerzální síť informací a dokumentace \autocite[246]{Rayward1994}. Uživatelé po celém světě by z této sítě získávali informace a případně do ní přispívali vlastními poznatky. Stálí zaměstnanci by obstarávali vyhledávání informací \autocite[191]{Wright2007}.

V Otletově vizi hraje hlavní roli univerzální, centralizovaný proces, který slouží, třídí a extrahuje fakta z dokumentů. Cílem je najít objektivní pravdu, a přestože jednotlivé dokumenty nemusí být pravdivé, není pochyby o tom, co je skutečně pravdivé. \Textcite[247]{Rayward1994} označuje Otletův pohled na vědomosti jako autoritářský, redukcionistický, positivistický, zjednodušující – a optimistický.

Otlet kladl velký důraz na hledání vazeb mezi informacemi, protože každý dokument dává smysl až v kontextu s dalšími informacemi: „Žádná bibliologická tvorba, bez ohledu na to, jak je mocná, nepředpokládá redistribuci, kombinaci a nová spojení“ \autocite[190]{Wright2007}.
A ačkoli Otlet neměl přímý vliv na rozvoj Webu, jak konstatuje \textcite[191]{Wright2007}, jeho vize sahala v některých ohledech dále, než současný Web s řadou technických omezení. Například problém fixních dokumentů, které si s sebou nesou vnořené odkazy jako vlastní vazby na další dokumenty. Teprve sofistikované vyhledávače dokážou odhalit širší vazby mezi dokumenty, avšak tyto informace zůstanou koncovému uživateli skryté, stejně tak jako historický význam dokumentu. Nejen v Otletově systému hrají odkazy mezi dokumenty mnohem větší roli, než jaká jim byla přisouzena v prostředí Webu.

\section{Memex}
\label{sec:memex}

Zatímco Paul Otlet rozvíjel svojí institucionální vizi organizace veškerého vědění, Vannevar Bush na druhé straně Atlantiku předpokládal podobné problémy s explozí informací. Řešení, ke kterému dospěl je ryze humanistické, individualistické a americké \autocite{Wright2015}.

Bush, jako vědec, inženýr a státní úředník, pozoroval prohlubující se specializaci jednotlivých vědeckých oborů a enormní nárůst množství informací. Zároveň vnímal, že prostředky pro práci s informacemi jsou zoufale zastaralé. Tyto problémy -- a návrh jejich řešení -- popisuje ve svém nejznámějším článku \citetitle*{Bush1945} publikovaném roku 1945 v magazínu \enterm{The~Atlantic Monthly}. Bush se zde zaměřuje zejména na nedostatky konvenčních hierarchických systémů třídění informací:

\begin{quoted}{\autocite[6]{Bush1945}}
%Our ineptitude in getting at the record is largely caused by the artificiality of systems of indexing. When data of any sort are placed in storage, they are filed alphabetically or numerically, and information is found (when it is) by tracing it down from subclass to subclass. It can be in only one place, unless duplicates are used; one has to have rules as to which path will locate it, and the rules are cumbersome. Having found one item, moreover, one has to emerge from the system and re-enter on a new path.
Naše neschopnost nalézt \textins{správný} záznam je způsobená zejména nepřirozeností indexovacích systémů. Když uložíme jakákoliv data, jsou založena abecedně nebo numericky a informace je nalezena (pokud je nalezena) přecházením od podtřídy k podtřídě. Záznam může být pouze na jednom místě, pokud neexistují duplikáty; jsou zapotřebí pravidla, jak jej lokalizovat a tato pravidla jsou těžkopádná. Navíc jakmile je nalezena jedna položka, je nutné se opět ze systému vynořit a vstoupit na novou cestu.
\end{quoted}

Proto navrhuje přizpůsobit uspořádání informací modelu lidského myšlení:

\begin{quoted}{\autocite[6]{Bush1945}}
%The human mind does not work that way. It operates by association. With one item in its grasp, it snaps instantly to the next that is suggested by the association of thoughts, in accordance with some intricate web of trails carried by the cells of the brain.
\textins{Lidská mysl} funguje na bázi asociace. Jakmile uchopí jednu položku, okamžitě přeskočí k další na základě asociace myšlenek na základě spletité sítě cest realizované mozkovými buňkami.
\end{quoted}

Ale cesty, které nejsou často používané, zmizí. Lidská mysl zapomíná a Bush navrhuje tento problém řešit pomocí technologie. V článku je popsané mechanické zařízení nazvané \enterm{Memex}: Pracovní stůl, ve kterém jsou veškeré materiály -- texty, knihy, noviny, dopisy -- uložené na svitcích mikrofilmu. To umožňuje kompresi velkého množství textů do malého prostoru: „I pokud by uživatel vložil pět tisíc stran materiálů denně, trvalo by mu stovky let, než zaplní celé úložiště“ \autocite[6]{Bush1945}. Další knihy si může uživatel objednat přímo na mikrofilmech. Na stole je několik obrazovek, na který se promítají jednotlivé stránky, klávesnice, páky a průhledná deska pro fotografování nových záznamů na mikrofilm Memexu. Pro vyvolání konkrétní knihy stačí zadat její kód a její titulní strana se okamžitě objeví na obrazovce. Pákou je pak možné listovat stránkami­ -- stačí zatlačit více a stránky se přetáčí po desítkách či po stovkách \autocite[6]{Bush1945}.

Co je však důležité, Memex umožňuje materiály spojovat. Uživatel si zobrazí dvě stránky vedle sebe na projektoru a stisknutím tlačítka mezi nimi vytvoří asociativní cestu (\enterm{associative trail}). Do cesty je možné přidávat další materiály, případně vlastní poznámky a anotace. Cestu je opět možné pomocí kódu vyvolat a případně jí zreprodukovat a sdílet \autocite[7]{Bush1945}. Bush předpokládá vznik nové formy literatury postavené na těchto cestách i profesi tzv. \enterm{trail blazers} kteří budou v masách informací hledat nové cesty.

Bush v návrhu Memexu zúročil dřívější zkušenosti s mechanickými počítači. V rámci civilního výzkumu na MIT ve dvacátých letech zkonstruoval zařízení pro výpočet diferenciálních rovnic \enterm{Differential Analyzer}. Analyzer, podobně jako většina mechanických počítačů fyzicky reflektuje úkol, který vykonává. Uživatel vidí celý proces a může se z něj i poučit \autocite[16--17]{Barnet2014}. Analyzer zároveň dokázal, že stroj dokáže automatizovat lidské kognitivní dovednosti \autocite[18]{Barnet2014}.  V roce 1935 byl Bush osloven americkým námořnictvem, aby vytvořil nejpokročilejší kryptoanalytický nástroj na světě. Nástroj by hledal společné výskyty písmen ve zprávách a pomohl tak rozluštit nové šifry používané mj. Japonskem. Bushův výzkumný tým, ve kterém pracoval i Claude Shannon, sestrojil \enterm{Comparator}. Ten využíval mikrofilm a fotoelektrický jev pro ukládání a porovnávání dat. Avšak mikrofilm se v praxi ukázal být příliš křehký. Proto tým přešel z původního návrhu na papírové pásky, které byly pomalé a nespolehlivé; celý projekt proto skončil neúspěchem. \autocite[18--19]{Barnet2014}

Navzdory prvotnímu neúspěchu s mikrofilmem Bush věřil, že se jedná o médium budoucnosti, které zjednoduší přístup k informacím. Nebyl sám -- H. G. Wells v roce 1938 psal o „Permanentní světové encyklopedii“ která by obsahovala veškeré vědění světa na mikrofilmu \autocite[19]{Barnet2014}. Bush proto použil mikrofilm i při návrhu zařízení \enterm{Rapid Selector}, které mělo modernizovat knihovnu. Abstrakty textů by byly převedeny na mikrofilm a role filmu jsou prohledávány přes fotosenzitivní přístroj \autocite[20]{Barnet2014}.

Jak Memex, tak Rapid Selector těžily z návrhu Comparatoru. Podobně jako Analyzer předtím, Comparator fyzicky a zcela transparentně automatizoval lidskou činnost. Technologie mikrofilmu navíc zapadala do tehdejší analogie o fungování mozku \autocite[18]{Barnet2014}. Memex je tak logickým rozšířením, mechanickým modelem lidské mysli.

\label{p:memex2}
Bushův původní popis Memexu inspiroval celou řadu dalších vynálezců a myslitelů, především Douglese Engelbarta a Teda Nelsona. Méně známé jsou však pozdější revize Memexu, které přináší mnohem radikálnější představy. Esej \citetitle*{Bush:Memex2} z roku 1959 byla publikovaná až po Bushově smrti \autocite{Bush:Memex2}. I zde Bush rozebírá zejména technické aspekty Memexu; namísto mikrofilmů navrhuje použití magnetických pásek, které nabízí mnohem lepší kompresi dat. Jako vrcholnou formu ukládání dat si představuje zápis do molekul organických krystalů. To by nabídlo dokonce kompaktnější uložení dat než lidský mozek. Dále se zabývá myšlenkou hlasového ovládání, které by obsloužilo jednoduché příkazy bez klávesnice. Memex získává i určité schopnosti personalizace a „učení se:“ dokáže zohlednit čas, který uživatel stráví na jednotlivých segmentech cesty materiálem. Memex může operovat i zcela autonomně a sám provádět po nocích komparativní analýzy mezi různými cestami. To zároveň předpokládá i vzdálený přístup k datům v Memexu. Bush spekuluje, že v knihovnách se budou nacházet „masivní Memexy“ s ohromnou databází dokumentů, ve kterých bude personál vytvářet nové cesty. Uživatelé budou moci přistupovat k těmto datům skrze telefonní linku a kopírovat materiál do svého osobního Memexu. A protože informace se lépe vstřebávají, když je člověk v pohodlí, Memex II. bude také obsahovat samostatnou obrazovku připojenou kabelem k hlavní jednotce, se kterou se může uvelebit v křesle. Obrazovka má vlastní ovládací prvky pro otáčení stránek a vkládání poznámek.

Bushův pozdější návrat k Memexu, text \citetitle*{Bush:MemexRev} publikovaný roku 1967 zmiňuje pouze část těchto nápadů \autocite{Bush:MemexRev}. Zohledňuje však pokrok na poli digitálních počítačů, ačkoli Bush preferuje cenovou dostupnost a snadnou údržbu analogových strojů. V šedesátých letech byly počítače vnímané především jako stroje pro zpracování dat, patřící mezi výsady velkých firem a univerzit. Analogový přístup naproti tomu nabízel transparentní a osvědčený způsob, jak simulovat model lidského myšlení \autocite[202]{Wright2007}.

% - redukce digitalizací (Wright?)

\label{p:memex:form}
Bush sdílí podobný záměr s Otletem v tom smyslu, že se snaží tradiční fyzickou formu knihy nahradit novou formou intertextuality, která by zjednodušila přímé odkazování mezi texty. To by umožnilo vyměnit tradiční externí a hierarchické systémy klasifikace za systém asociací, který by mohli vytvářet i sami čtenáři \autocite[199]{Wright2007}. Avšak zatímco Otlet chtěl asociaci mezi texty použít především k vylepšení hierarchického systému třídění, Bush chtěl tradiční hierarchické systémy zcela nahradit \autocite[199--200]{Wright2007}. V původním návrhu Memexu Bush naopak zachovává formu dokumentů, kde stránky dokumentů tvoří základní stavební kameny asociací a systém se nedostane k samotnému obsahu textu \autocite[200]{Wright2007}. Memex II. sice počítá s možností asociace referencí ke konkrétním odstavcům textu \autocite[175]{Bush:Memex2}, ale i magnetické pásky zde slouží spíše jako robustnější alternativa mikrofilmu, kde jednotlivé stránky jsou zaznamenané podobně, jako snímky filmu. Jak v Bushově, tak v Otletově konceptu se nejspíše zrcadlí profese jejich autorů: Bush se jako technik soustředí především na formu a technickou funkčnost systému, zatímco Otlet, jakožto knihovník, se zaměřuje na proces extrakce a organizace informací z textů.

Memex nikdy nebyl postavený a Bush si byl vědom, že v době publikace původního článku nebylo možné jej realizovat \autocite[69]{Barnet2008}. Paradoxně však díky tomu Bushův koncept přežil do dnešních dnů; Memex se stal „obrazem potenciálu“ – teoretickým konstruktem a vodítkem, které nadále zůstává zdrojem inspirace \autocite[200]{Wright2007}.

% For many of the hypertext pioneers, but most obviously Bush and Nelson, conventional methods of indexing are disastrous, and association is more ‘natural’, more human. \autocite[23]{Barnet2014}

% Yet Bush’s vision remains sur-prisingly misunderstood, even by many of those who claim to embrace it. “It is strange that [‘As We May Think’] has been taken so to heart in the field of information retrieval,” writes Ted Nelson, “since it runs counter to virtually all work being pursued under the name of information retrieval today.”7 Indeed, Bush’s essay in some ways reads as an indictment of everything that was about to go wrong with the computer industry. \autocite[193]{Wright2007}

\section{NLS/Augment}
\label{sec:nls}

V létě roku 1945 pracoval dvacetiletý Douglas Engelbart jako vojenský radarový technik na Filipínách. Jednoho dne zavítal do knihovny Červeného kříže kde si přečetl Bushovu esej. Koncept zařízení, které by podpořilo kognitivní schopnosti člověka Engelbarta nesmírně zaujal. Údajně několik dní potom každému na potkání vyprávěl, o čem se dočetl \autocite[7]{Markoff2005}. \textquote[{\autocite[45]{Barnet2014}}]{Některé nápady jsou jako semínka. Nebo jako virus. Když jsou ve vzduchu ve správnou dobu, infikují přesně ty, kteří jsou nejnáchylnější položit svůj život pro realizaci nápadu}. Ačkoli si to tehdy ještě neuvědomoval, Engelbart byl myšlenkou Memexu infikován.

O pět let později se Engelbart zasnoubil a náhle si uvědomil, že dosáhl všeho, čeho v životě dosáhnout chtěl. Proto začal systematicky zvažovat, čemu dalšímu by mohl věnovat úsilí. Náhle v sérii záblesků spatřil vizi, jak by si lidstvo mohlo poradit s narůstající komplexitou všech činností \autocite[9]{Markoff2005}. Tuto vizi formuloval ve třech „záblescích,“ na kterých postavil své celoživotní úsilí \autocite[45]{Barnet2014}:

\begin{quote}
\begin{description}
%\item[FLASH-1]

\item[ZÁBLESK-1]
\enquote{Komplexita problémů lidstva narůstá rychleji, než naše schopnosti je zvládat.}
%The difficulty of mankind’s problems was increasing at a greater rate than our ability to cope.

\item[ZÁBLESK-2]
\enquote{Podpoření schopnosti lidstva zvládat komplexní, akutní problémy by mohla být zajímavá oblast, ve které by mladý člověk mohl přinést největší změnu.} %TODO 
%Boosting mankind’s ability to cope with complex, urgent problems would be an attractive candidate as an arena in which a young person might ‘make the most difference’.

\item[ZÁBLESK-3]
\enquote{Ahá -- přede mnou se formuje barvitá představa, jak sedím před velkou CRT konzolí a pracuji novými způsoby, které se rychle rozvíjí před mým zrakem (počínaje vzpomínkami na radarové obrazovky, které jsem dříve opravoval).} % TODO
%Ahah-graphic vision surges forth of me sitting at a large CRT console, working in ways that are rapidly evolving in front of my eyes (beginning from memories of the radar-screen consoles I used to service)
\end{description}
\sourceatright{\autocites[189]{Engelbart1988}[cit. podle][45]{Barnet2014}}
\end{quote}

Myšlenka interaktivní obrazovky připojené k počítači byla původním východiskem Engelbartova projektu pro „zvýšení lidského intelektu.“\footnote{Respektive „augmentace“ z původního \enterm{augmenting human intellect}.} Engelbart si však byl vědom, že naše schopnost řešit problémy má dva aspekty. Externí pomůcky, technologie jako psací nástroje, papír, telefon atp. tvoří „systém nástrojů“ (\enterm{Tool System}). Kognitivní, smyslově motorický aparát, techniky jako jazyk, dovednosti, paradigmata, postupy aj. tvoří „lidský systém.“ Společně tvoří „systém augmentace“ (\enterm{Augmentation System}), který nám umožňuje fungovat ve světě a je podmíněný prostředím i kulturou \autocite{DEI2013}. Oba systémy se vzájemně doplňují: lidský systém nás „vybavuje“ pro práci s nástroji a nástroje naopak ovlivňují v rekurzivní smyčce vývoj lidského systému \autocite[38]{Barnet2014}; dochází ke „koevoluci“ obou systémů. Engelbartovým cílem proto bylo tuto koevoluci řízeně urychlit; testovací subjekty tvořil Engelbartův výzkumný tým \autocite{DEI2013}. Ten si tak fakticky zlepšoval svou vlastní dovednost řešit problémy, což vedlo k urychlení celého výzkumného procesu; Engelbart tento fenomén pojmenoval \enterm{bootstrapping}\footnote{U vývoje software se používá termín \enterm{dogfooding}, kdy vývojáři sami používají produkt, který vyvíjí. Většina produktů ale není vyvíjena za účelem podpory svého vývoje, proto se nejedná o totéž.}.

Engelbart demonstroval myšlenky adaptace na nástroje a koevoluce skrze „deaugmentaci“ schopností. Připevnil cihlu k tužce a měřil rychlost psaní v porovnání s obyčejnou tužkou a psacím strojem \autocites[47]{Markoff2005}[\pno~2c4m]{Engelbart1962}. Pochopitelně psaní s cihlou bylo pomalé a omezující. Obdobně i technologie, se kterými pracujeme neslouží pouze k vyjádření myšlenek, ale aktivně vymezují limity našeho myšlení \autocite[52]{Barnet2014}.

V rámci prvotního rozvoje konceptu financovaného americkým letectvem Engelbart sepsal svou vizi v textu \citetitle*{Engelbart1962}, publikovaném v roce 1962. Zde je popsán komplexní technický systém, který pomáhá řešit „profesní problémy diplomatů, vedoucích pracovníků, sociologů, vědců, fyziků, advokátů, projektantů -- ať už problém existuje dvacet minut, nebo dvacet let“ \autocite[\pno~1a1]{Engelbart1962}. Na příkladu architekta popisuje interakci s počítačem (uživatelovým „koncipientem“), vybaveným displejem, klávesnicí a „ukazatelem“ \autocite[\ppno~1a11--12]{Engelbart1962}.

% \begin{quote}
% By "augmenting human intellect" we mean increasing the capability of a man to approach a complex problem situation, to gain comprehension to suit his particular needs, and to derive solutions to problems. Increased capability in this respect is taken to mean a mixture of the following: more-rapid comprehension, better comprehension, the possibility of gaining a useful degree of comprehension in a situation that previously was too complex, speedier solutions, better solutions, and the possibility of finding solutions to problems that before seemed insoluble. And by "complex situations" we include the professional problems of diplomats, executives, social scientists, life scientists, physical scientists, attorneys, designers--whether the problem situation exists for twenty minutes or twenty years. We do not speak of isolated clever tricks that help in particular situations. We refer to a way of life in an integrated domain where hunches, cut-and-try, intangibles, and the human "feel for a situation" usefully co-exist with powerful concepts, streamlined terminology and notation, sophisticated methods, and high-powered electronic aids.
% „Zvýšením lidského intelektu“ myslíme zlepšení schopnosti člověka přistoupit ke složité problémové situaci, získat porozumění pro jeho specifickou potřebu a odvodit řešení problémů. [...] ...„složitou situací“ myslíme profesní problémy diplomatů, vedoucích pracovníků, sociologů, vědců, fyziků, advokátů, projektantů -- ať už problémová situace existuje dvacet minut, nebo dvacet let. Nemluvíme o samostatných chytrých tricích které pomohou v konkrétních situacích. Popisujeme způsob žití v integrované doméně, ve které předtuchy, pokus a omyl, neurčitosti a lidský „pocit ze situace“ smysluplně koexistuje se silnými koncepty, zefektivněnou terminologií a notací, sofistikovanými postupy a výkonnými elektrickými pomůckami.
% \sourceatright{\autocite[1a1]{Engelbart1962}}
% \end{quote}

Engelbartovy myšlenky však vzbudily odpor. Počítače neměly obrazovky; byly to velké, drahé stroje, ke kterým uživatelé neměli přístup. Dávkové úlohy na zpracování dat se předávaly technikům na děrných štítcích. Engelbart navrhoval věnovat drahocenný výpočetní čas individuálním uživatelům. Zároveň jeho vize nezapadala do existujících paradigmat \autocite[46]{Barnet2014}.

Po několika letech vývoje na Stanfordském výzkumném institutu (SRI) však Engelbart předvedl v prosinci roku 1968 funkční prototyp \enterm{oN-Line System} (NLS) na konferenci \enterm{Fall Joint Computer Conference} v San Franciscu. Zde Engelbart světu poprvé předvedl interaktivní grafické rozhraní ovládané myší s podporou rozdělování obrazovky, asociativním odkazováním a videokonferencí \autocite[60]{Barnet2014}. Událost nyní označovaná jako \enterm{Mother of All Demos} byla zrodem nové éry počítačů \autocite[61]{Barnet2014}.

NLS je hierarchický systém jehož hlavní funkcí je \enterm{Journal}. Uživatel zde vkládá a zanořuje položky, každá má nadpis, text omezené délky a hierarchické pořadové číslo (například \texttt{1a} pro první položku druhé úrovně, \texttt{1a1b} pro druhou položku čtvrté úrovně atp.).\footnote{Současnou obdobou jsou aplikace nazývané \enterm{outliner}.}
Položky je možné mezi sebou odkazovat skrze pořadová čísla, která však nejsou stabilní. Proto NLS umožňuje adresovat položky i pomocí permanentních identifikátorů (každá položka má unikátní identifikátor v rámci souboru) a relativních odkazů \autocite[16--17]{Muller-Prove2002}. Odkazy nejsou součástí textu, ale ukládají se separátně \autocite[43]{Barnet2014}. Tato vlastnost bude k vidění i u jiných hypertextových systémů (včetně Xanadu) a zásadně se liší od koncepce odkazování v HTML, kde odkazy tvoří permanentní součást dokumentu.
Hierarchická povaha systému umožnila vytvořit sdílenou strukturu, ve které členové týmu udržovali veškeré materiály projektu, včetně zdrojového kódu a e-mailových konverzací. Podstromy je možné nezávisle měnit a zobrazovat různými pohledy a „listy“ stromu si udržují přehled změn \autocite[42]{Barnet2014}.

„Vedlejším efektem“ vývoje NLS byl vynález řady technik softwarového inženýrství. Protože NLS obsahoval enormní množství kódu, programátoři zjistili, že potřebují sledovat historie změn, hlášení chyb a změnové požadavky. Postupy, které jsou dnes běžnou součástí vývoje software tehdy neexistovaly, tým kolem NLS se vše učil ze samotného procesu vývoje \autocite[57--58]{Barnet2014}.

Avšak během několika let po demonstraci systému se vývoj NLS zpomalil a řada klíčových členů týmu projekt opustila. Důvodem mohla být změna zaměření -- NLS byl po technické stránce hotový, v souladu s konceptem koevoluce byla řada na lidském systému. To však nebylo pro Engelbartův tým tolik zajímavé, jako vývoj nových technologií \autocite[62]{Barnet2014}. V roce 1969 vzniklo výzkumné centrum Xerox PARC, do kterého si část Engelbartova týmu odnesla klíčové myšlenky z NLS.%
\footnote{Odliv zaměstnanců byl tak velký, že si v \enterm{Augmentation Research Center}, kde se NLS vyvíjel, začali ironicky přezdívat jako \enterm{Xerox Research Training Center} \autocite[202]{Markoff2005}.}
Zaměstnanci PARC později vytvořili dominantní paradigma grafického uživatelského rozhraní postaveného na oknech, ikonách a nabídkách (WIMP) \autocite[61]{Barnet2014}.

O nasazení systému NLS byl zájem ze strany komerčních firem. Systém však byl velice komplexní a složitý na ovládání. Uživatelé se museli naučit stovky příkazů \autocite[200]{Markoff2005}, firmy ale měly zájem pouze o malou část funkcí systému \autocite[62]{Barnet2014}. Uživatelské rozhraní NLS nebylo „přirozené“ -- nicméně pro Engelbarta není technologie přirozená, každá technická dovednost je získaná a lidé se musí učit z technologie \autocite[40]{Barnet2014}.

Engelbart na základě holistického pohledu na symbiózu člověka a techniky přenesl koncept Memexu do digitálního světa. Podařilo se mu demonstrovat praktické přínosy počítače jako nástroje pro tvorbu i spolupráci. NLS představoval jednotný systém pro více uživatelů, kde informace nejsou uzavřené v konkrétní aplikaci nebo souboru. V NLS je možné odkazy provázat libovolnou informaci, což představuje milník hypertextových systémů, kterému se někteří nástupci přiblížili, ale nepodařilo se jej překonat.

\section{HES a FRESS}
\label{sec:hes}

Andries „Andy“ van Dam je známou osobností zejména na poli počítačové grafiky. Je jedním z autorů přelomové učebnice \enterm{Computer Graphics: Principles and Practice} a založil skupinu ACM SICGRAPH\footnote{Předchůdce ACM SIGGRAPH, která pořádá stejnojmennou konferenci.} \autocite[91]{Barnet2014}.
% Mezi jeho bývalé studenty patří významní členové studia Pixar \autocite[91]{Barnet2014}.
Van Dam se taktéž zasloužil o vznik praktické výuky počítačové vědy na Brownově univerzitě, kde také vedl vývoj dvou ranných hypertextových systémů HES a FRESS.

V roce 1967 potkal van Dam svého bývalého spolužáka ze Swarthmore College Teda Nelsona. Ten tehdy rozvíjel svou představu hypertextového systému Xanadu. Nelson měl vizi, avšak neměl žádné ukázky ani prototypy. Van Dam mu proto nabídl spolupráci a dal dohromady vývojářský tým studentů. Záhy se však ukázalo, že vize van Dama a Nelsona se rozchází. Van Dam zdůrazňuje, že cílem nebylo implementovat celou vizi Xanadu, ale pouze některé Nelsonovy koncepty \autocite[99--101]{Barnet2014}. Projekt měl dva cíle: interaktivní vytváření dokumentů pro tisk a průzkum konceptu hypertextu \autocite[889]{vanDam1988}.

Projekt dostal název \enterm{Hypertext Editing System} (HES) a podle van Dama šlo o první interaktivní (\enterm{online}) textový procesor pro komerčně dostupné počítače. Fungoval na běžně dostupném mainframu od IBM, zatímco NLS vyžadoval proprietární konfiguraci a hardware \autocite[104--105]{Barnet2014}. Uživatel vybíral text pomocí světelného pera a mohl jej měnit, kopírovat, přesouvat, vyjímat a vkládat. Dokumenty se skládaly ze série textových rámců, které (na rozdíl od položek v NLS) neměly pevnou strukturu a omezenou délku. Van Dam i Nelson kladli ve svých systémech velký důraz na minimální limitaci autorů \autocite[103]{Barnet2014}. Textové rámce byly propojené třemi způsoby: jednosměrnými hypertextovými odkazy, které byly vizualizované jako hvězdička v textu a vedly k jinému rámci jako volitelná cesta textem; větvemi, které tvořily menu, kde si uživatel mohl vybrat cestu k dalšímu rámci \autocite[104]{Barnet2014} a neviditelnými „spoji“ (\enterm{splices}), které určovaly, jak se má hypertext linearizovat pro tisk \autocite[889]{vanDam1988}. Mimo to HES umožňoval vkládat textové \textit{instance}, které odkazovaly na stejný textový rámec, takže změny v rámci se projevily ve všech jeho instancích; van Dam k tomu podotýká: „Instance jsou běžná myšlenka v počítačové grafice -- nic velkého“ \autocite[889]{vanDam1988}. Obdobně funguje i princip \textit{transkluze} v systému Xanadu \autocite[18]{Muller-Prove2002}. Pro usnadnění navigace v hypertextu Nelson taktéž navrhl implementaci tlačítka zpět, které po následování odkazu uživatele vrátilo na předchozí rámec \autocite[104]{Barnet2014}.

Systém byl orientovaný na netechnické uživatele, kteří produkovali a editovali text. Van Dam, podobně jako Engelbart, však měl problém myšlenku interaktivní tvorby na počítači prosadit. Když v roce 1968 HES prezentoval redakci Time/Life, dozvěděl se, že bude trvat alespoň deset let, než budou novináři ochotní si sednout k monitoru \autocite[890]{vanDam1988}. Pro prezentaci systému vymyslel van Dam techniku \enterm{progressive disclosure}, tj. postupné odhalování funkcí. Standardní IBM klávesnice měla 32 funkčních kláves, což laické uživatele děsilo. Van Dam proto vyrobil plastové krytky, které ponechaly odhalených jen pět kláves pro manipulaci s textem. Jak si uživatelé systém postupně zkoušeli, byly jim odhalovány další řady funkcí \autocite[890]{vanDam1988}. O úspěchu HES hovoří fakt, že jej využila NASA pro tvorbu dokumentace k vesmírnému programu Apollo \autocites[889]{vanDam1988}[106]{Barnet2014}.

Nelson je dodnes ze svého působení na Brownově univerzitě zklamaný. Z jeho hlediska se van Damův tým příliš soustředil na zpracování tiskových dokumentů -- „simulaci papíru“; funkce hypertextu byly výrazně zjednodušené a omezené \autocite[100]{Barnet2014}. Podle Nelsona dal HES světu omezenou a špatnou představu o hypertextu, která vedla i k podobě dnešních webových prohlížečů. Van Dam i \textcite[107]{Barnet2014} pochybují, že by HES tak významně ovlivnil evoluci dalších systémů.

%https://commons.wikimedia.org/wiki/File:HypertextEditingSystemConsoleBrownUniv1969.jpg

% NLS -> FRESS

V roce 1968 van Dam viděl Engelbartovu demonstraci NLS. HES v té době už dosahoval svých limitů jako jednouživatelský systém a van Dam uvažoval o nové verzi. NLS ukázal van Damovi výhody víceuživatelského systému, nezávislosti na výstupním zařízení a funkcí „outlineru.“ \autocites[890]{vanDam1988}[61]{Barnet2014} Poté, co van Dam strávil den studiem NLS v Engelbartově laboratoři, usoudil, že to „byla nejlepší věc, jakou kdy viděl“ \autocite[158]{Markoff2005}.

Van Damův tým studentů spojil silné stránky HES a NLS do nového systému nazvaného \enterm{File Retrieval and Editing System} (FRESS). Zatímco HES byl navržený pro specifický model terminálu, FRESS fungoval na široké škále terminálů s různými vstupními i zobrazovacími zařízeními \autocites[218]{Wright2007}[108]{Barnet2014}. K systému mohlo přistupovat více uživatelů, ačkoli nepodporoval sdílenou editaci dokumentů v reálném čase jako NLS \autocite[108]{Barnet2014}. Na druhou stranu FRESS podporoval funkci \enterm{undo}, vrácení úpravy textu o jeden krok zpět -- vůbec první implementaci této funkce v textovém editoru \autocites[891]{vanDam1988}[108]{Barnet2014}.

FRESS nabízel sofistikovanější hypertextové funkce. Podporoval dva druhy odkazů: jednosměrné „tagy,“ které ukazovaly na jednoduché elementy (typicky anotace, poznámky pod čarou) a obousměrné „skoky“ (\enterm{jumps}) mezi dokumenty, které byly v cílovém dokumentu viditelné ve formě zpětných odkazů. Vazby mezi dokumenty bylo možné vizualizovat ve formě mapy \autocite[109]{Barnet2014}. Protože myš ještě nebyla běžným vybavením počítačů, uživatelé se navigovali mezi odkazy pomocí světelného pera a pedálu -- „\enterm{point--and--kick}“ \autocites[19]{Muller-Prove2002}[218]{Wright2007}. Odkazy mohly vést na libovolný segment textu v dokumentu \autocite[109]{Barnet2014} a zároveň se ukládaly v samostatné databázi, částečně nezávisle na obsahu dokumentu. To umožnilo uživatelům postupně odkrývat odkazy mezi dokumenty a vytvářet vlastní anotace bez zásahu do originálního textu.

Funkce FRESS byly cílené především na praktické využití elektronického textu a hypertextu ve výuce. Brownova univerzita nabízela dva experimentální kurzy ve kterých veškerá četba, tvorba a diskuze mimo hodiny probíhala „online“. V porovnání s kontrolními skupinami byli studenti v těchto kurzech výrazně aktivnější, více se zapojovali a dosahovali lepších známek \autocite[219]{Wright2007}.
Po skončení kurzu byl v systému zachovaný hypertextový „korpus“, který skupina za semestr vytvořila \autocite[219]{Wright2007}. Například studenti kurzu anglické poezie nejprve samostatně anotovali texty a postupně získávali přístup k anotacím svých spolužáků a nakonec celé „komunální“ databázi anotací a odkazů k originálním textům -- van Dam to nazývá „elektronickým graffiti“ \autocite[891]{vanDam1988}.

FRESS ukázal význam počítačů a hypertextu pro humanitní vědy. Systém se využíval pro výuku na Brownově univerzitě během sedmdesátých let a našel si cestu i na univerzity v Nizozemí \autocite[110]{Barnet2014}. Další van Damův projekt, \enterm{Electronic Document System} (EDS) se zaměřil na grafickou vizualizaci dokumentů a interaktivní funkce -- značil posun k hypermédiím \autocite[111]{Barnet2014}. V roce 1983 vznikl na Brownově univerzitě \enterm{Institute for Research in Information and Scholarship} (IRIS) jehož nejvýznamějším projektem byl systém Intermedia \autocite[220]{Wright2007}.

\section{Intermedia}
\label{sec:intermedia}

Projekt Intermedia začal vznikat na Brownově univerzitě pod IRIS v roce 1985 a od začátku sloužil pro podporu výuky -- některé výukové materiály z tohoto systému se zachovaly dodnes. Například hypertext \enterm{The Victorian Web} původně vytvořil George Landow v systému Intermedia, později byl přenesen do Storyspace a na Web, kde je nadále rozvíjen\footnote{Viz \url{http://www.victorianweb.org/}} \autocite[112]{Barnet2014}.

Na rozdíl od svých předchůdců není Intermedia pouze samostatná aplikace pro tvorbu propojených dokumentů. Jedná se o integrované prostředí, které poskytuje jednotnou podporu pro vytváření odkazů mezi aplikacemi \autocite[51]{Nielsen1995}. Balíček aplikací v Intermedia zahrnoval mj. textový editor \enterm{InterWord} (resp. \enterm{InterText}), grafický editor \enterm{InterDraw}, \enterm{InterSpect} pro zobrazování 3D grafiky, \enterm{InterPlay} pro animace, přehrávač videa \enterm{InterVideo}, e-mailový klient \enterm{InterMail} a nástroj \enterm{InterVal} pro tvorbu časových os \autocites[82]{Yankelovich1988}{Haan1992}[29]{Muller-Prove2002}.

Tvůrci Intermedia se snažili integrovat hypermediální funkce do systému takovým způsobem, aby byly stejně univerzální a dostupné jako funkce „kopírovat a vložit“ v operačních systémech Macintosh a Windows \autocite[38]{Haan1992}. Proto každá z aplikací Intermedia podporuje stejnou funkcionalitu vytváření anotací a odkazů, které propojují objekty napříč soubory. Uživatel v jedné aplikaci vybere objekt nebo fragment textu, z nabídky vybere funkci „Začít odkaz“ a v jiné aplikaci odkaz dokončí nad jiným objektem \autocite[82]{Yankelovich1988}. Odkazy mezi aplikacemi jsou ukládané do samostatné sdílené databáze, jsou obousměrné a mohou obsahovat metadata (například informace o autorovi nebo datu vytvoření, nebo popisek). Odkazy jsou organizované v „sítích“ (\enterm{webs}), které mohou uživatelé společně vytvářet a sdílet. Stejný dokument tak může hrát různé role v různých sítích -- uživatelé si mohou zobrazit sítě, které je zajímají \autocite[30]{Muller-Prove2002}.

Separace obsahu od odkazů má praktický důsledek pro práci s materiály ve výuce:

\begin{quoted}{\autocites[27]{Yankelovich1985}[112]{Barnet2014}}
% For exam- ple, the English Department might have a web referencing all of the Shakespearean tragedies along with links pertaining to color imagery in those plays, while the Religious Studies Department might have a web refer- encing those same plays with links per- taining to religious symbolism.
Například ústav anglického jazyka může mít síť všech Shakespearových tragédií s odkazy na metaforičnost barev v těchto hrách, zatímco ústav religionistiky může mít síť těch samých her s odkazy na náboženskou symboliku.
\end{quoted}

Stejný koncept může být využitý i k archivaci jednotlivých verzí sítě \autocite[30]{Muller-Prove2002}. Uživatelé mohou mít různé úrovně oprávnění k jednotlivým dokumentům; vedle běžných práv pro čtení a zápis je to i oprávnění vytvářet anotace \autocite[90]{Yankelovich1988}.

Systém Intermedia zaznamenal v akademických kruzích úspěch, avšak na začátku devadesátých let se jeho vývoj zastavil. Software Intermedia byl úzce provázaný s konkrétní verzí operačního systému A/UX, Unixové verze Mac~OS. Nová verze A/UX přinesla řadu zpětně nekompatibilních změn, které software rozbily a IRIS nezískal zdroje potřebné pro další vývoj \autocites[51]{Nielsen1995}[112--113]{Barnet2014}[221]{Wright2007}. Uživatelé přenesli řadu materiálů do systému Storyspace a koncept Intermedia našel svého pokračovatele v projektu Microcosm.

\section{Storyspace}
\label{sec:storyspace}

Michael Joyce je spisovatel a profesor anglického jazyka známý zejména svými hypertextovými romány. Když začátkem osmdesátých let prováděl poslední revizi nové knihy, uvědomil si, že by se text na jedné straně výborně hodil na začátku knihy. Vzápětí mu však došlo, že „ve skutečnosti chtěl udělat něco jiného“:

\begin{quoted}{\autocite[31]{Joyce:WIR}}
%What I really wanted to do, I discovered, was not merely to move a paragraph from page 265 to page 7 but to do so almost endlessly. I wanted, quite simply, to write a novel that would change in successive readings and to make those changing versions according to the connections that I had for some time naturally discovered in the process of writing and that I wanted my readers to share.
Co jsem ve skutečnosti chtěl udělat nebylo jen přesunout odstavec ze strany 265 na stranu 7 ale dělat to takřka do nekonečna. Jednoduše jsem chtěl napsat román který by se změnil s každým čtením a tyto měnící se verze by vnikaly na základě vazeb, které jsem průběžně objevil během psaní a o které jsem se chtěl podělit se čtenáři.
Z mého pohledu mohou odstavce na mnoha různých stránkách následovat po odstavcích na mnoha jiných stránkách, ačkoli s rozdílnými výsledky a s různými záměry. \textelp{}
Přišlo mi, že pokud já, jako autor, můžu použít počítač k přeházení odstavců, nebylo by obtížné nechat čtenáře dělat totéž podle nějakého schématu, které jsem určil předem. Navíc jsem věděl, že mí studenti literární tvorby by měli užitek z možnosti rozpoznávat a měnit vnitřní vazby ve svých textech.
\end{quoted}

Ve spolupráci s Jay D. Bolterem vytvořil aplikaci \enterm{Storyspace} ve které Joyce napsal jednu z prvních „hyperfikcí“ -- \enterm{afternoon, a~story} \autocite[115--116]{Barnet2014}. Jak Joyce popisuje v eseji \citetitle*{Joyce:WIR} o počátcích Storyspace, základní funkce programu vznikly, aniž by tvůrci znali pojem hypertext \autocite[32]{Joyce:WIR}. Postupně se však dostali do kontaktu s tradicí studia hypermédií, která \textquote[{\autocite[32]{Joyce:WIR}}]{jako chobotnice kouká jedním okem nad vodou, ale pod hladinou její chapadla sahají téměř všude, včetně pedagogiky, lingvistiky, kognitivních věd, literatury, fyziky, databázové teorie, klasické literatury, mediálních studií, medicíny a tak dále. Protože se tato tradice zabývá odkazy a souvislostmi, nezná žádných intelektuálních hranic}.

\label{p:storyspace:cell}
Storyspace vizualizuje strukturu hypertextu ve dvourozměrném prostoru. Kusy informací jsou organizované do buněk, tzv. „prostorů pro psaní“ (\enterm{writing spaces}), které obsahují titulek, text a další buňky \autocites[382]{Joyce1991}[29]{Muller-Prove2002}. Díky tomu Storyspace umožňuje vytvářet hierarchie a buňky rekurzivně organizovat. Do buněk lze psát pomocí textového editoru s podporou obrázků; text je možné následně „explodovat“ do samostatných buněk pro reorganizaci \autocite[383]{Joyce1991}. Podobně jako u FRESS a Intermedia lze odkazy tvořit mezi jednotlivými fragmenty textu, případně je možné se odkázat na buňku nebo i jiné dokumenty. Buňky mohou být uspořádané do cest, které mohou sloužit například k linearizaci textu pro tisk nebo nabízet různé tematické průchody dokumentem \autocite[385]{Joyce1991}.

Specifickou vlastností Storyspace je explicitní oddělení rolí tvůrce a čtenáře. Čtenář nevidí celý korpus hypertextu a vazby mezi buňkami, namísto toho text postupně prochází a vytváří si vlastní výklad. \Textcite[219--220]{Ryan2001} používá metaforu kaleidoskopu ve kterém zabudovaná zrcadla zaručují líbivý dojem bez ohledu na uspořádání tvarů. Obdobně je možné jednotlivé fragmenty textu uspořádat do nejrůznějších kombinací, které však musí vytvářet podobně atraktivní a smysluplný čtenářský zážitek. Storyspace proto nabízí i funkci tzv. \enterm{guard fields}, které zpřístupní čtenáři odkaz jen po splnění určité podmínky -- typicky po průchodu určitými buňkami \autocites[386]{Joyce1991}[125]{Barnet2014}. 

Storyspace po třiceti letech nadále vyvíjí a prodává společnost Eastgate Systems \autocite{Eastgate:Storyspace}. Podobně jako většina raných hypertextových systémů vznikl Storyspace jako samostatná aplikace, bez podpory síťového propojení nebo spolupráce více uživatelů \autocite[131,134]{Barnet2014}. Jedná se však o užitečný nástroj pro tvorbu lineárních i nelineárních textů, kde hypertextová funkcionalita může být využitá pro elektronickou publikaci textu, nebo jako prostředek autora k utřídění si myšlenek.

\section{Microcosm}
\label{sec:microcosm}

Projekt Microcosm začal v roce 1989 na Southamptonské univerzitě s cílem umožnit propojení napříč nejrůznějšími aplikacemi. Autoři zvolili podobnou metodu jako systém Intermedia: odkazy jsou samostatné entity uložené v databázích (\enterm{linkbases}) \autocite[31]{Muller-Prove2002} a hypertextové funkce jsou dostupné jako služba operačního systému.
Microcosm však kromě vlastních aplikací nabízí integraci s existujícími aplikacemi (jako např. Microsoft Word) a portabilitu mezi operačními systémy.

Systém umožňuje vytvářet několik druhů odkazů mezi dokumenty: „Specifické odkazy“ umožňují propojit zcela konkrétní objekt nebo fragment  textu. „Lokální odkazy“ mezi sebou propojují konkrétní dokumenty jako celky. „Generické odkazy“ pak umožňují propojit libovolný objekt v libovolných dokumentech s konkrétní destinací -- příkladem může být slovník odborných termínů: systém může v novém dokumentu automaticky vytvořit odkazy do slovníku\footnote{Jak si všímá \textcite[31]{Muller-Prove2002}, tato funkcionalita je zobecněním odkazů v systému HyperTIES. Obdobou je i automatické vytváření odkazů pomocí \enterm{CamelCase} v některých wiki systémech.}
\autocites[184]{Davis1992}[4]{Fountain1990}. Zajímavá je funkce „dynamických odkazů,“ které nemají konkrétní zdroj ani cíl -- namísto toho jsou výsledky generované algoritmicky, například hledáním podobnosti mezi dokumenty \autocite[31]{Muller-Prove2002}.

Microcosm rozlišuje tři typy aplikací (\enterm{viewers}) podle míry integrace se systémem (\enterm{awareness}): Plně integrované aplikace přímo od autorů Microcosm podporují všechny funkce systému. Částečně integrované aplikace byly adaptované pomocí rozšíření a umožňují systému předat informace pro vytvoření odkazů. Neintegrované aplikace je možné podporovat alespoň skrze systémovou schránku (tj. funkce kopírovat a vložit). V případě horší integrace není možné uživatele odkázat na konkrétní část dokumentu, Microcosm v takovém případě podporuje alespoň lokální a generické odkazy \autocite[185--186]{Davis1992}.

Nevýhodou Microcosm je riziko zneplatnění odkazů, které vyplývá z jejich separace od samotných dokumentů. Pokud je dokument smazaný nebo upravený v aplikaci, která není do systému integrovaná, odkazy se rozbijí. Obdobný je problém s verzováním: pokud se odkazovaný dokument změnil, není jednoznačné, zda by měl uživatel dostat původní nebo novou verzi \autocite[188--189]{Davis1992}.

Projekt Microcosm se pokoušel integrovat funkce hypertextu do operačního systému s využitím existujících aplikací. Snahou bylo otevřít dokumenty pro propojování a odkazování. V tomto záměru se společně se systémy Intermedia a Hyper-G řadí mezi „otevřené hypermediální systémy“ (\enterm{Open Hypermedia Systems}) \autocite[43]{Muller-Prove2002}. 

\section{Hyper-G/HyperWave}
\label{sec:hyperg}

Systém Hyper-G vzniknul v polovině devadesátých let jako „síťový informační systém druhé generace“ který řeší nedostatky systémů první generace, tj. Gopher a WWW. Autoři z Technické univerzity ve Štýrském hradci (TU Graz) se v článku \citetitle*{Andrews1995} zaměřují na šest nedostatků Webu a navrhují jejich řešení \autocite{Andrews1995}.

Web poskytuje pouze jednosměrné odkazy, které jsou pevnou součástí textových dokumentů; typicky se dokumenty přesouvají nebo se mažou, což vede k nefunkčním odkazům \autocite[207]{Andrews1995}. Server pro Hyper-G si udržuje samostatnou databázi odkazů, kterou automaticky synchronizuje s dalšími servery \autocites[208]{Andrews1995}[35]{Muller-Prove2002}. 

Web nenabízí žádné nativní nástroje pro vyhledávání, je zcela závislý na externích vyhledávacích službách které mají omezené pokrytí a neposkytují aktuální výsledky \autocite[207]{Andrews1995}. Hyper-G má zabudované funkce pro vyhledávání a je možné vyhledávat na několika serverech paralelně \autocites[208]{Andrews1995}.

Pro implementaci pokročilých funkcí je zapotřebí logika na serveru, která narušuje uniformitu rozhraní -- každý Webový server se chová jinak, \textquote[{\autocite[207]{Andrews1995}}]{což vede k \enquote{balkanizaci} (jak říká Ted Nelson) Webu do nezávislých \enquote{W3 impérií}}. Tím, že Hyper-G nabízí v jádru více funkcí pro získávání a strukturování informací, není zřejmě tyto funkce nutné implementovat samostatně pro každou stránku. Hyper-G má i zabudovaný model autentizace a oprávnění \autocite[208]{Andrews1995}.

\label{p:hyperg:collection}
S pokročilými funkcemi souvisí i nedostatečná podpora pro správu většího množství dat na Webu \autocite[207]{Andrews1995}. Hyper-G umožňuje soubory organizovat do kolekcí, které je možné do sebe vnořovat a jeden dokument může náležet do více kolekcí. Speciálním typem kolekce je \enterm{cluster}, který kombinuje související soubory do jedné logické jednotky -- například překlady stejného dokumentu nebo video se související transkripcí. Zároveň je možné se na kolekce odkazovat jako na celek \autocite[36]{Muller-Prove2002}.

Dalším problémem Webu je fakt, že stránky jsou pouze pro čtení; uživatelé mohou informace pouze pasivně procházet \autocite[207]{Andrews1995}. Klientská aplikace pro Hyper-G, prohlížeč Harmony má v sobě zabudované funkce pro editaci dokumentů a protokol \enterm{Harmony Document Viewer Protocol} (DVP) umožňuje změny v dokumentech ukládat na serveru \autocite[212]{Andrews1995}.

V poslední řadě je  Web obtížně škálovatelný -- populární stránky mohou být zahlcené uživateli \autocite[207]{Andrews1995}. Server Hyper-G umí automaticky replikovat a aktualizovat obsah, což usnadňuje rozložení zátěže \autocites[10]{Andrews1995}[38]{Muller-Prove2002}.

Adopci Hyper-G měla pomoci kompatibilita s existujícími systémy. K Hyper-G serveru mohou, kromě nativních klientů, přistupovat i prohlížeče pro Web a Gopher. Server samotný se chová jako klient pro tyto protokoly, což mu umožňuje získat data z existujících serverů. V systémech „první generace“ prohlížeče přistupují ke každému serveru samostatně. Naproti tomu v Hyper-G se uživatel připojuje ke „svému“ Hyper-G serveru, který zajišťuje autentizaci a může si ukládat data ze vzdálených zdrojů pro rychlejší přístup \autocite[211]{Andrews1995}. Klientské aplikace pro Hyper-G proto mohou podporovat pouze nativní protokol, podpora pro jiné systémy je řešená na serveru.

Prohlížeč Harmony zahrnuje podporu pro řadu formátů, včetně videí a 3D scén. Podobně jako v Intermedia může uživatel vytvářet odkazy mezi sebou, včetně anotování videí \autocite[213]{Andrews1995} pochopitelně s tou výhodou, že vše je možné snadno publikovat přes Internet. Harmony zároveň vytváří navigační mapu, která uživateli ukazuje jeho cestu hypertextem \autocite[214]{Andrews1995}. Harmony nebyl jediným prohlížečem, další implementace i podrobný popis celého systému rozebírá kniha \citetitle*{Maurer1996} která je volně dostupná v elektronické verzi na stránkách TU Graz \autocite{Maurer1996}.

Hyper-G svojí sofistikovaností velice připomíná systém Xanadu. Autoři vytvořili i komerční verzi systému pojmenovanou HyperWave. Nicméně i přes svou robustnost a kompatibilitu s dosavadními systémy se systém Hyper-G neujal. Web i přes veškeré nedostatky kritizované autory Hyper-G měl jednu zásadní výhodu: adaptabilitu (\pnoref[viz třetí kapitola,]{sec:xanavsweb}).

\section{Enquire a World Wide Web}
\label{sec:www}

Tim Berners-Lee začal pracovat v laboratořích CERN v roce 1980 jako konzultant. Od začátku jej fascinovala složitost organizační hierarchie a heterogenita prostředí v CERN, proto si vytvořil aplikaci Enquire, která mu pomáhala sledovat vazby mezi lidmi, programy a hardwarem. Informace v Enquire jsou organizované na „kartičkách,“ které mezi sebou mají obousměrné sémantické vazby, jako „A vyrobil B“ nebo „A je součástí B“ \autocite{Berners-Lee1994}. 

Během svého pozdějšího působení na CERN ve druhé polovině osmdesátých let viděl potřebu jednotného informačního systému podobného Enquire ale pro celý CERN. Existující systémy nebyly vyhovující, protože uživatele nutily organizovat informace specifickým způsobem \autocite[15]{Berners-Lee1999}. Berners-Lee proto přišel v roce 1989 s návrhem na nový informační systém, postavený na principech hypertextu -- zejména možnosti odkazovat se na zdroje \autocite{Berners-Lee1989}. Hlavní motivací byl problém ztrácejících se informací a nedostupné dokumentace v důsledku velké fluktuace zaměstnanců -- nicméně „CERN je miniaturou zbytku světa za několik let“ \autocite[\pno~\enterm{Losing Information at~CERN}]{Berners-Lee1989}.
Cílem bylo vytvořit systém dostatečně obecný a zároveň přizpůsobivý pro různé případy použití \autocite[20]{Berners-Lee1999}.

\label{p:web:compat}
Berners-Lee v návrhu apeluje zejména na kompatibilitu a interoperabilitu navrhovaného systému. Musí fungovat napříč různými operačními systémy, musí podporovat různá zobrazovací zařízení (základem jsou textové terminály, grafika může být řešená až časem)%
\footnote{Současným ekvivalentem tohoto konceptu je \enterm{progressive enhancement}, \textcite[viz][]{Gustafson2008}.} %
a být decentralizovaný -- nikdo nemusí žádat o povolení k jeho nasazení \autocites[\pno~\enterm{CERN Requirements}]{Berners-Lee1989}[16]{Berners-Lee1999}. Zároveň umožní přístup k aktuálním datům uloženým v existujících systémech, ať už jsou to databáze nebo manuálové stránky; odkaz v dokumentu může případně spustit  aplikaci \autocite[\pno~\enterm{Accessing Existing Data}]{Berners-Lee1989}.

Systém \enterm{World Wide Web} je definován třemi elementy (v pořadí důležitosti): adresní schéma (\enterm{Universal} či \enterm{Uniform Resource Identifier} -- URI), protokol pro přenos dat (\enterm{HyperText Transfer Protocol} -- HTTP) a značkovací jazyk (\enterm{HyperText Markup Language} -- HTML). Berners-Lee měl problém předat myšlenku, že \enterm{World Wide Web} nezahrnuje nic víc -- neexistuje uzavřená síť kde tyto technologie fungují, ani centrální autorita, která Web provozuje; Web samotný je „informační prostor“ \autocite[36]{Berners-Lee1999}. Univerzální adresní schéma zajišťuje, že je možné získat libovolnou informaci z „prostoru.“ HTTP následně umožňuje, aby se klient a server „domluvili“ na přenosu informace ve formátu, se kterým oba umí pracovat \autocite[36--37]{Berners-Lee1999} -- to podporuje princip interoperability. HTML dokumenty obsahují odkazy na další zdroje, proto není zapotřebí žádná centrální databáze odkazů \autocite[16]{Berners-Lee1999}.

\label{p:web:live}
Řada klíčových rozhodnutí v návrhu Webu je vnímána jako zásadní slabiny systému -- viz např. Hyper-G (\pnoref{sec:hyperg}).
Přístup k „živým datům“ znamená, že informace z Webu mohou beze stopy zmizet, nebo se zcela změnit. Odkazy, které jsou pevnou součástí dokumentů znemožňují, tentýž text použít v jiném kontextu. Web se také soustředí pouze na propojitelnost textu, hypermédia, jako zvuk a video, explicitně nebyly součástí návrhu \autocites{Berners-Lee1990}{Berners-Lee1989}. V původním kontextu Webu, jako informačního systému pro CERN se však jednalo o logické ústupky.

\label{p:web:edit}
Berners-Lee v návrhu Webu počítal s funkcemi, které bychom v současných prohlížečích hledali marně. Web měl usnadnit jak přístup k informacím, tak jejich publikaci. Proto první webový prohlížeč (\enterm{WorldWideWeb}, později přejmenovaný na \enterm{Nexus}) v sobě zahrnoval vizuální editor HTML dokumentů. Dokumenty pak bylo možné přímo z prohlížeče publikovat na serveru pomocí HTTP metody POST \autocite[34]{Muller-Prove2002}. Vedle editace dokumentů je v návrhu zmíněná možnost „přidávání soukromých anotací“ \autocite[\pno~\enterm{Private links}]{Berners-Lee1989}. Prohlížeč Mosaic, který se zasloužil o popularizaci Webu, měl podporu pro vytváření skupinových anotací, která však byla v raných verzích odstraněná \autocite{Summers2013}. 

\label{p:web:semantic}
Berners-Lee v původním návrhu Webu také popisuje možnosti strojového zpracování hypertextové databáze skrze „odkazy s typy“ \autocites[\pno~\enterm{Data analysis}]{Berners-Lee1989}[21]{Berners-Lee1999}. Zárodky tohoto nápadu jsou viditelné v aplikaci Enquire, ve které karty nejsou propojené prostými odkazy, ale mají mezi sebou sémantické vazby. Berners-Lee tuto myšlenku později rozvedl ve vizi „Sémantického webu“ tj. zpřístupnění znalosti na Webu skrze strojově čitelná data. To by umožnilo softwarovým agentům „porozumět“ datům pro automatizaci některých činností nebo sofistikovanější získávání informací \autocite[177--198]{Berners-Lee1999}. To evokuje představu Vannevara Bushe o autonomním Memexu (\pnoref{p:memex2}). Web však nebyl navržený tak, aby čtenáři mohli snadno vytvářet asociativních vazby mezi dokumenty -- jedinou možností je vytvořit zcela nový dokument, který tyto vazby popíše.

%https://www.w3.org/History/1994/WWW/Journals/CACM/screensnap2_24c.gif


\section{Shrnutí}

V kapitole byly představeny některé hypertextové systémy, jejich záměry a motivace jejich tvůrců. Na systémech je patrné, že koncept hypertextu byl pojímán mnohem šířeji, než jak jej demonstruje současný Web. Zároveň se konkrétní aplikace hypertextu vyvíjely společně s rozvojem technologií. Paul Otlet se ve svém díle snaží „osvobodit“ informace z jejich fyzických forem v knihách a přenáší je na kartičky. Memex Vannevara Bushe stále operuje v mantinelech fyzických dokumentů a stránek, byť redukovaných na mikrofilmy. Douglas Engelbart přenáší vizi Memexu na digitální počítače. Výsledný systém NLS tvoří unifikované prostředí, kde je hypertext integrovaná funkce systému. Obdobně systémy HES a FRESS Andries van Dama pracují s~ucelenou hypertextovou databází. Intermedia značí posun do roviny multimédií a snahu vytvořit obdobně jednotný systém pro hypermédia. Microcosm pak znamená posun k pragmatismu: protože aplikací a datových formátů je příliš mnoho, systém se snaží vytvořit hypermediální vrstvu nad konvenčními operačními systémy. World Wide Web pak problém hypertextu přenáší do síťového, distribuovaného systému -- zároveň však rezignuje na problematiku užší propojitelnosti dokumentů a hypermédií, stačí, že jsou zdroje dostupné. Tento a další problémy Webu se snaží řešit systémy jako Hyper-G, avšak za~cenu větší komplexity.

Web ve své jednoduchosti a decentralizovanosti sice do velké míry naplnil představu o jednotném informačním prostoru, avšak za cenu velkých kompromisů. Uživatel Webu je automaticky redukován do role konzumenta, publikace obsahu není implicitní a univerzální součástí Webu \autocite[220]{Glut}, ale funkce specializovaných nástrojů nebo netriviální technická dovednost. Současný Web není „dokuverzum“, jaké si představoval Ted Nelson, kde je možné vytvářet a remixovat stejně snadno jako číst a kde se žádná verze žádného dokumentu nikdy neztratí. Tuto vizi popisuje následující kapitola.

% Second, Ted talks a lot about the docuverse, a mythi-and contains
% cal entity out there that is all-inclusive
% But instead, right now we are building
% everything.
% docu-islands; none of our systems talk to each other,
% they {are wholly incompatible. So we are all working
% the same agenda, more or less, but we can’t exchange
% stuff; there is no exchange format, there is no univer-our systems are closed systems.
% sality, and furthermore,
% In a sense, they are making the same mistake as the all-in personal computing. Yes, they
% in-arm environments
% give you a word processor and a spreadsheet editor and
% a business graphics package, etc., but none of them are
% really satisfactory. And our experience with FRESS,
% where we had to escape to command language, showed
% to be able to go outside. So it’s
% that it is really important
% not enough to bundle the HyperCard package with
% every Mac you buy. It really ought to be migrated
% down, become part of the toolbox, so that application
% programmers can take their applications and take ad-vantage of a standard linking protocol that works
% within and between applications.
% [897]{vanDam1988}
